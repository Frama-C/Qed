\documentclass[twocolumn]{article}
\usepackage[utf8]{inputenc}
\usepackage[T1]{fontenc}
\usepackage[english]{babel}
\usepackage{amsmath}
\usepackage{listings}
\usepackage[colorlinks]{hyperref}
\usepackage{xspace}
\usepackage{multicol}
\usepackage{abstract}
\setlength{\parindent}{0pt}
\setlength{\parskip}{\baselineskip}
%% --- Names
\newcommand{\Qed}{\textsf{Qed}\xspace}
\newcommand{\C}{\textsf{C}\xspace}
\newcommand{\ACSL}{\textsf{ACSL}\xspace}
\newcommand{\Ocaml}{\textsf{Ocaml}\xspace}
\newcommand{\Coq}{\textsf{Coq}\xspace}
\newcommand{\Why}{\textsf{Why}\xspace}
\newcommand{\Isabelle}{\textsf{Why}\xspace}
\newcommand{\FramaC}{\textsf{\mbox{Frama-C}}\xspace}

%% --- Maths
\newcommand{\prop}{\mathtt{prop}}
\newcommand{\bool}{\mathtt{bool}}
\newcommand{\true}{\mathtt{true}}
\newcommand{\false}{\mathtt{false}}
\newcommand{\X}{X}
\newcommand{\Z}{\mathbf{Z}}
\newcommand{\R}{\mathbf{R}}
\newenvironment{bnf}{\begin{array}[t]{c@{\,}c@{\,}l}}{\end{array}}
\newcommand{\limply}{\rightarrow}
\newcommand{\sem}[1]{[\![\,#1\,]\!]}

%% --- Dummy
\newcommand{\TODO}{\textcolor{red}{[\ldots]}}


\begin{document}
\title{\Qed: proving in logic modulo}
\author{Loïc Correnson}
\twocolumn[
\maketitle

\begin{abstract}
  We propose a framework for manipulating in a efficient way terms and
  formul{\ae} in classical logic modulo equality. \Qed was firstly
  designed for the generation of proof obligations of a
  weakest-precondition engine for \C programs inside the \FramaC
  framework. Efficiency is obtained by on-the-fly strong normalization
  of terms and maximal sharing of sub-terms. Then, \Qed is equipped by
  an extensible simplification engine. Rewriting rules,
  abstract-interpretation over terms and decision procedures are
  experimented inside the simplification engine. We finally present
  both an automated and interactive theorem prover in logic modulo.
\end{abstract}
\vspace{1cm}
]

\section*{Introduction}

Automated theorem prover for classical logic is an extensive research
domain with very powerful operational results. Not all theorem provers
are efficient over the same class of properties, and it is interesting
to combine several theorem provers to achieve proof of complex
logic propositions.

\TODO

\section{Logic Modulo}
\label{logic}

We present here the fundamental logic system behind \Qed. It is a
simply typed logic system, using natural signed integers $z\in\Z$,
real numbers $r\in\R$ and user-defined abstract datatypes as ground
terms. We first introduce the (abstract) syntax for the language, and
then its typing rules.

\subsection{The language}

To simplify notations, we do not separate \emph{predicates} from
\emph{terms} at the syntax level, but at the typing level.  Actually,
such a design also simplifies the implementation of normalization and
maximal sharing and eliminates the redundancy between boolean and
predicate operators. 

We denote $\kappa$ the category of terms, which consists of type
$\prop$ for predicates, and $\tau$ for terms. We classically restrict
existential and universal quantification to terms. We denote $\true$
and $\false$ the overloaded truth values of type $\bool$ and $\prop$.

Types also include polymorphic abstract data types, that we call \emph{sorts} and denote $s\in{\cal S}$. For instance, we write $\mathtt{list}(\Z)$ for lists over integers.

Terms can be constructed by application of functions. Each function $f\in\cal F$ is associated to a \emph{signature} $\sem{f}$ which is a partial application from $\tau^*$ to $\tau$. Variables are denoted $x\in\cal X$.

The complete logical language is depicted on figure~\ref{logic-lang}.

\begin{figure}[tp]
  \[
  \begin{array}{|c|}
  \hline\\
  \begin{bnf}
    \kappa &::=& \prop \;|\;\tau \\
    \tau &::=& \\ 
      &|& \bool \\
      &|& \Z \\
      &|& \R \\
      &|& s(\tau^*) \\
    \\
    X &::=& <a:\tau>
  \end{bnf}
  \quad
  \begin{bnf}
    t &::=& \\
    &|& x\in\X \\
    &|& \true \,|\, \false \\
    &|& z\in\Z \,|\, r\in\R \\
    &|& f(t^*) \\
    &|& t \;\mathtt{binop}\; t \\
    &|& \mathtt{unop}\;t \\
    &|& \Pi x.\,t \\
  \end{bnf}
  \\\\
  \begin{bnf}
    \mathtt{binop} &::=&
    \begin{array}[t]{c}
      +, -, \times, \mathtt{div}, \mathtt{mod} \\
      <, \leq, =, \neq \\
      \land, \lor, \limply
    \end{array} \\
    \mathtt{unop} &::=& -, \lnot \\
        \Pi &::=& \forall,\exists,\lambda
  \end{bnf}
  \\
  \\
  \hline
  \end{array}
  \]
  \caption{Types and Terms}
  \label{logic-lang}
\end{figure}

\subsection{Well formed terms}

Terms must be correctly typed in a very standard way. Boolean values can be promoted to propositions, and logical connectors are overloaded over booleans and propositions. Arithmetic operators are overloaded on $\Z$ and $\R$ with promotion from integer to real when necessary. This induces a partial order $(\sqsubseteq)$ over types, and we denote $(\sqcup)$ the associated least upper bound when it is defined:
\[
\begin{array}{ccc}
    \bool\sqsubseteq\prop & 
    \Z\sqsubseteq\R  \\
    \bool\sqcup\prop = \prop &
    \Z\sqcup\R = \R
\end{array}
\]

Partial order over arithmetic types is \emph{not} lifted to polymorphic datatypes, since it would induce a kind of recursive transformation to promote sparse integers into real, which can be difficult to define for arbitrary types.

\begin{figure}[p]
  \[
  \TODO
  \]
  \caption{Typing Rules for Terms}
  \label{typing-terms}
\end{figure}

\section{Normalization of Terms}
\label{terms}


\section{Proving Theorems}
\label{proof}




\end{document}
